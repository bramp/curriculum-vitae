\documentclass[a4paper,10pt]{article}

%A Few Useful Packages
\usepackage{marvosym}
\usepackage{fontspec} 					%for loading fonts
\usepackage{xunicode,xltxtra,url,parskip} 	%other packages for formatting
\RequirePackage{color,graphicx}
\usepackage[usenames,dvipsnames]{xcolor}
%\usepackage[big]{layaureo} 				%better formatting of the A4 page
\usepackage{fullpage}
% an alternative to Layaureo can be ** \usepackage{fullpage} **
\usepackage{supertabular} 				%for Grades
\usepackage{titlesec}					%custom \section

%Setup hyperref package, and colours for links
\usepackage{hyperref}
\definecolor{linkcolour}{rgb}{0,0.2,0.6}
\hypersetup{colorlinks,breaklinks,urlcolor=linkcolour, linkcolor=linkcolour}

%FONTS
\defaultfontfeatures{Mapping=tex-text}
%\setmainfont[SmallCapsFont = Fontin SmallCaps]{Fontin}

%CV Sections inspired by: 
%http://stefano.italians.nl/archives/26
\titleformat{\section}{\Large\scshape\raggedright}{}{0em}{}[\titlerule]
\titlespacing{\section}{0pt}{3pt}{3pt}
%Tweak a bit the top margin
%\addtolength{\voffset}{-1.3cm}

%Italian hyphenation for the word: ''corporations''
%\hyphenation{im-pre-se}

%-------------WATERMARK TEST [**not part of a CV**]---------------
\usepackage[absolute]{textpos}

\setlength{\TPHorizModule}{30mm}
\setlength{\TPVertModule}{\TPHorizModule}
\textblockorigin{2mm}{0.65\paperheight}
\setlength{\parindent}{0pt}

%--------------------BEGIN DOCUMENT----------------------
\begin{document}

%WATERMARK TEST [**not part of a CV**]---------------
\font\wm=''Baskerville:color=787878'' at 8pt
\font\wmweb=''Baskerville:color=FF1493'' at 8pt
{\wm 
	\begin{textblock}{1}(0,0)
		\rotatebox{-90}{\parbox{500mm}{
			Typeset by Alessandro Plasmati with \XeTeX\  \today\ for 
			{\wmweb \href{http://www.aleplasmati.comuv.com}{aleplasmati.comuv.com}}
		}
	}
	\end{textblock}
}

\pagestyle{empty} % non-numbered pages

\font\fb=''[cmr10]'' %for use with \LaTeX command

%--------------------TITLE-------------
\par{\centering
		{\Huge Andrew \textsc{Brampton}
	}\bigskip\par}

%--------------------SECTIONS-----------------------------------
%Section: Personal Data
\section{Personal Data}

\begin{tabular}{rl}
    \textsc{Date of Birth:} & 18 November 1982 (27 years old) \\
    \textsc{Address:}   & 1-3 DeVitre Street, Lancaster, LA1 1QU \\
    \textsc{Phone:}     & +44 7821 158879\\
    \textsc{Email:}     & \href{mailto:me@bramp.net}{me@bramp.net}\\
    \textsc{Website:}   & \href{mailto:http://bramp.net}{http://bramp.net}\\
\end{tabular}

%Section: Education
\section{Education}
\begin{tabular}{rl}	
 \textsc{2008} & \textbf{Ph.D. in Computer Science, Lancaster University, UK}\\
& Thesis: ``The Impact of Highly Interactive Workloads on Video-on-Demand Systems''\\
& Supervisor: Prof. Laurent Mathy\\
& Available at: \href{http://bramp.net/thesis}{http://bramp.net/thesis}\\
& Research Interests: Content Distribution Networks, Peer-to-Peer, Autonomic Self-Organising Systems\\
&\\

\textsc{2004} & \textbf{B.Sc. (Hons) in Computer Science, Lancaster University, UK}\\
& A first class honours degree\\
& Dissertation: ``Peer-to-Peer Media Streaming''\\
& Supervisor: Nicholas Race\\
&\\

\textsc{2001} & \textbf{A-Levels, Hind-Leys Community College, Leicestershire, UK}\\
& in Computer Science, Mathematics, Further Mathematics, Physics and General Studies.\\

\end{tabular}

%Section: Work Experience at the top
\section{Work History}
\begin{tabular}{r|p{11cm}}
 \emph{Current} & Research Associate at \textsc{Lancaster University} \\
 \textsc{Sept 2007} & \emph{Networking Researcher}\\
&\footnotesize During the three years at Lancaster I have been involved in numerous projects, some of which I worked on my own, and others where I worked in small teams. My tasks would typically be research oritated (such as running and analysing experiments, and publishing my results), but in additional to this I have helped with teaching undergraduate and masters students. This included running seminars, lab supervising, marking, and lecturing.\\

%\multicolumn{2}{c}{} \\

\end{tabular}

\section{Projects}
 \textbf{Stealth Distributed Hash Table}\\
% Java simulator
% C++ implementation that ran over a unrelaible planet sized testbed
% Many publications
 {\footnotesize The beginning of my Ph.D. focused on distrubuted hash tables (DHTs), specifically how to make them more reliable and suitable for high performance applications. This work led to me create a Java based peer-to-peer simulator with implementations of a couple of DHTs, followed by a C++ implementation suitable for running real experiments over PlanetLab. During this time I learnt many techiniques for making fast and high performing Java applications, as well as learning the perils of trying to use a 650 node network scattered across the globe as a experimental testbed. This work led to numerous conference papers and a couple of journal publications, as well as source code being made available.}

 \textbf{Interactive Video-on-Demand}\\
% PHP website + tools to aid in capturing, preparing and streaming
 \footnotesize{Whilst working on the Autonomic Content Distribution Network (aCDN) project I wanted to understand the workloads generated by users seeking heavily though a video, and how I could create algorithms and techniques to improve this. To begin this work I created a Video-on-Demand platform which could be viewed via a website. Then I captured and transcoded the 2006 FIFA world cup (soccer) and made these matches available to users on the Lancaster Campus. This generated enough data for me to form statistical models, which led on to algorithms that a server can use to aid in prefetching and caching to improve the user's experience. This generated a few conference papers and was the basis for my Ph.D. thesis.}

 \textbf{10Gbit+ Networking on Multi-core platforms}\\
% Hacking the Linux kernel
 \footnotesize{This project aimed to improve general networking on multi-process or multi-core servers whilst using 10Gbit or more networks. To achieve this we highly instrimented the Linux kernel, which allowed us to figure out exactly what the problems are with high speed networking, and specifically what impact multi-core archecitures had. Later we developed a few techniques specifically designed for multi-core environments which would improve performance for multiple applications. This work produced a couple of conference papers and later became a chapter in a coleages Ph.D. thesis.}

 \textbf{High performance routing on commodity hardware}\\
% Hacking the FreeBSD kernel
 \footnotesize{Over the last two years in my free time I've been involved in a technology start-up whose aim is to create a high performance networking platform running on low cost commodity hardware. To this end I have been in chief architect and developer creating a flexible system able to sustain high packet throughputs. This work is based on a custom networking engine written as a FreeBSD kernel module. Over the two years I have had extensive experience developing for the FreeBSD 7 and 8 kernels, and have managed to achieve near linear scaling of packet forwarding with the number of cores, up to a tested rate of 40Mpps. This networking platform has not yet been publicly announced, and its our hope to get funding in the near future.}

 \textbf{Open Source}\\
 \footnotesize{In my free time I have contributed to numerous Open Source projects, and have even started a few myself. To name just a few I have had patches accepted by, Linux, FreeBSD, PHP, Google Chrome, Google Android, Intel's Networking Drivers, HeidiSQL, and many more. In addition to this I have open sourced and made available numerous projects ranging from academic network simulation and benchmarking to an application to mount Nintendo DS Roms, to scripts for converting ordnance survey coordinates to GPS coordinates. }

\section{Publications}
\begin{tabular}{rl}

\end{tabular}


%Section: Scholarships and additional info
%\section{Scholarships and Certificates}
%\begin{tabular}{rl}
% 2004 & Scholarship for graduate students with an outstanding curriculum \footnotesize(\EURcr 30,000)\normalsize\\
% \textsc{June} 2006 & {\textsc{Gmat}\textregistered}\setmainfont[SmallCapsFont=Fontin SmallCaps]{Fontin-Regular}: 730 (\textsc{q:50;v:39}) 96\textsuperscript{th} percentile; \textsc{awa}: 6.0/6.0 (89\textsuperscript{th} percentile)
%\end{tabular}

\section{Computer Skills}
\begin{tabular}{rp{12cm}}

\multicolumn{2}{l}{\textbf{Programming}} \\
Advanced & C++, C, Java, PHP and \LaTeX \\
Casual   & Bash, Python, Pascal, C\# and .Net, ASP, x86 ASM, and other miscellaneous languages \\
Notes    & Experienced developer, with strong OO knowledge, with skills ranging from Kernel hacking, application development, and scripting. Has designed and implemented large scale distributed systems and robust applications that require a high availability\\

\\
\multicolumn{2}{l}{\textbf{Platforms}} \\
Advanced       & Windows XP, Vista, Server\\
               & Linux (specifically Debian) and FreeBSD 4 and above\\
Above Average  & Configuring and running Apache+PHP, MySQL, OpenLDAP, PostgresSQL, and other services\\

\\
\multicolumn{2}{l}{\textbf{Simulations}} \\
Advanced & Designing and implementing a network simulator\\
Good     & Using PlanetLab (a planetary scale test bed)\\
Casual   & Running NS-2, OMNeT++ and other simulators\\

\\
\multicolumn{2}{l}{\textbf{Web Technologies}} \\
Advanced & HTML, CSS, JavaScript, PHP, CGI, AJAX, and numerous frameworks such as JQuery, Kohana and Zend Framework.\\

\\
\multicolumn{2}{l}{\textbf{Mathematical}} \\
Good & Matlab, Maple, R, gnuplot\\

\end{tabular}

\section{Interests and Activities}
Technology, Open-Source, Programming\\
Paradoxes in Decision Making, Psychoanalysis, Behavioural Finance\\
Football, Travelling

\vfill{}

\begin{center}
{\scriptsize  Last updated: \today\- •\- 
Typeset in \XeLaTeX\\
\href{http://bramp.net/cv}{http://bramp.net/cv}}
\end{center}

\end{document}
